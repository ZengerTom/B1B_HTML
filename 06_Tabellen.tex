 \chapter{Tabellen}
Eine Tabelle in HTML dient nicht der Formatierung sondern nur der logischen Anordnung der Daten. Die Formatierung erfolgt wie üblich über CSS.
\section{einfache Tabellenstruktur}
Eine einfache Tabellenstruktur besteht aus folgenden Elementen:
\begin{itemize}
\item Tabelle \texttt{<table>}
\item Reihe \texttt{<tr>}
\item Kopfzeile \texttt{<th>}
\item Zelle \texttt{<td>}
\end{itemize}
\begin{lstlisting}[caption=''einfache Tabelle'']
<table>
   <tr>   <th>H1</th>    <th>H2</th>    <th>H3</th>    </tr>
   <tr>   <td>C11</td>   <td>C12</td>   <td>C13</td>   </tr>
   <tr>   <td>C21</td>   <td>C22</td>   <td>C23</td>   </tr>
</table>
\end{lstlisting}
\section{Attribute}
In HTML5 wurden fast alle Tabellenattribute entfernt. Lediglich folgende Eigenschaften stehen zur Verfügung:
\begin{tabbing}
xxxxxxxxxxxxxxxxxxx\=\kill
<table>		\>border=''1'' (Rahmen wird angezeigt)\\
<td> \& <th>	\>colspan\\
			\>rowspan\\
			\>scope=''col'' (Überschrift für Spalte)\\
			\>scope=''row'' (Überschrift für Zeile)\\
\end{tabbing}
\subsection{Attribute <colspan> \& <rowspan>}
Mit Hilfe der Attribute \texttt{<colspan>} und \texttt{<rowspan>} können Spalten bzw. Zeilen zusammengefasst werden. Der übergebene Zahlenwert des Attributs beschreibt die Anzahl der Zellen.
\section{Struktur}
Zur Strukturierung der Tabelle stehen folgende drei Elemente zur Verfügung:
\begin{itemize}
\item Kopfbereich \texttt{<thead>}
\item Datenbereich \texttt{<tbody>}
\item Fußbereich \texttt{<tfoot>}
\end{itemize}
\section{Gruppierung}
Mit den Elementen \texttt{<colgroup>} \& \texttt{<col>} können einzelne Bereiche einer Tabelle ausgezeichnet werden. Die Auszeichnung muss direkt hinter dem öffnenden \texttt{<table>} Element erfolgen. Spaltenübergriffe können mit dem Attribut \texttt{span} erreicht werden.
\section{Beschriftung}
\subsection{Element <caption>}
Nach dem öffnenden \texttt{<table>} Tag erfolgt die Beschriftung der Tabelle mittels des \texttt{<caption>} Elements. Es kann nur eine Beschriftung pro Tabelle verwendet werden.
\subsection{Element <figcaption>}
Die Verwendung erfolgt äquivalent zu allen anderen Elementen, welche in ein \texttt{<figure>} Elternelement verpackt werden können.
