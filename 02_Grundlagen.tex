\chapter{Einführung und Grundlagen} 
\section{Grundgerüst}
\begin{lstlisting}[caption=''HTML Grundgerüst'']
<!doctype html>
<html>
  <head>
    <meta charset="UTF-8">
    <title>Fenstertitel</title>
  </head>
  <body>
    <h1>Hauptuberschrift</h1>
    <p>Text</p>
  </body>
</html>
\end{lstlisting}
\section{Dokumententyp}
Erste Angabe im Dokument. Definiert den Typ des Dokuments. Bei älteren Dokumententypen muss eine Dokumenttypdefinition (DTD) erfolgen.\\[-1.5em]
\begin{tabbing}
xxxxxxxxxx\=\kill
HTML5	\>\verb=<!doctype html>=\\
\end{tabbing}
\section{Wurzelelement <html>}
Nach der Definition des Doctypes folgt das Wurzelelement \texttt{<html>}. Dies umschließt alle anderen HTML Elemente. Dem Wurzelelement kann ein Sprachattribut mitgegeben werden (\verb@lang=“de“@).
\section{Grammatik}
Groß- und Kleinschreibung findet in HTML keine Beachtung. Allerdings ist auf einen einheitlichen Stil zu achten.
\subsection*{HTML Elemente}
Ein HTML Element besteht aus einem öffnenden und einem schließenden Tag. Zugehörige Inhalt wird von diesen Tags eingeschlossen.
\subsection*{Standalone Elemente}
Alleinstehende Tags haben keinen Inhalt und benötigen deswegen keinen schließenden Tag.
\subsection*{Optionale Tags}
Manche Elemente erlauben ein Weglassen des Ende Tags.
\subsection*{Verschachtelung}
HTML Elemente können ineinander verschachtelt werden. Auf ein korrektes schließen der Tags ist zu achten.
\subsection*{Attribute}
Den Elementen können verschiedene passende Attribute zugeordnet werden. Folgende Arten von Attributen können zugewiesen werden:
\begin{itemize}
\item Attribut mit fester Wertzuweisung
\item Attribut mit freier Wertzuweisung
\item Attribut mit freier Wertzuweisung ohne Konvention
\item allein stehende Attribute
\end{itemize}
\section{Kommentar}
\begin{lstlisting}[caption=''Kommentare in HTML'']
< ! - - Kommentarzeile - - >

< ! - - 
	Kommentar
	Kommentar
- - >
\end{lstlisting}
\section{DOM - Document Object Model}
Ein HTML-Dokument kann als Baumstruktur interpretiert werden. Jedes Objekt wird dabei als Knoten dargestellt und kann mit Hilfe von z.B. JavaScript manipuliert werden.



